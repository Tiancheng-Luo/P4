%   P4 (Polynomial Planar Phase Portraits) GUI SOURCE CODE
%   Software to study polynomial planar differential systems and represent
%   their phase portrait in several spaces, such as Poincaré sphere. 
%   URL: http://github.com/oscarsaleta/P4-src-gui
%   
%   Copyright (C) 1996-2016  J.C. Artés, C. Herssens, P. De Maesschalck,
%                            F. Dumortier, J. Llibre, O. Saleta
%   
%   This program is free software: you can redistribute it and/or modify
%   it under the terms of the GNU Lesser General Public License as published
%   by the Free Software Foundation, either version 3 of the License, or
%   (at your option) any later version.
%   
%   This program is distributed in the hope that it will be useful,
%   but WITHOUT ANY WARRANTY; without even the implied warranty of
%   MERCHANTABILITY or FITNESS FOR A PARTICULAR PURPOSE.  See the
%   GNU Lesser General Public License for more details.
%   
%   You should have received a copy of the GNU Lesser General Public License
%   along with this program.  If not, see <http://www.gnu.org/licenses/>.
%      
%
% TO BE INCLUDED IN THE PREAMBLE OF THE TEX FILE WITH \INPUT{MAPLEDEF}.
%
%

\definecolor{maplebgcolor}{rgb}{0.97,.97,.97}

\lstdefinelanguage{maple}
    {%
    morestring=[b]",
    morecomment=[l]\#,
    sensitive=false,
    morekeywords=[1]{proc,local,global,end,for,from,to,by,do,%
            od,if,else,fi,and,or,then,not,restart,type,return,%
    minus,union,read,save,set,list,member},
    morekeywords=[2]{simplify,normal,expand,expanded,evalf,map,collect,diff,%
            subs,eval,op,nops,coeff,rational,float,even,odd,convert,degree,sign,(,),[,]}}

\lstset{%
    basicstyle=\small,
    keywordstyle=[1]\bfseries,
    keywordstyle=[2]\underbar,
    identifierstyle=\slshape,
    firstnumber=auto,
    numbers=left,
    commentstyle=,
    stringstyle=\ttfamily,
    showstringspaces=false,
    language=maple,
    backgroundcolor=\color{maplebgcolor},
    flexiblecolumns=true
}
