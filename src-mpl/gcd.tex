%  This file is part of P4
% 
%  Copyright (C) 1996-2017  J.C. Artés, P. De Maesschalck, F. Dumortier,
%                           C. Herssens, J. Llibre, O. Saleta, J. Torregrosa
% 
%  P4 is free software: you can redistribute it and/or modify
%  it under the terms of the GNU Lesser General Public License as published
%  by the Free Software Foundation, either version 3 of the License, or
%  (at your option) any later version.
% 
%  This program is distributed in the hope that it will be useful,
%  but WITHOUT ANY WARRANTY; without even the implied warranty of
%  MERCHANTABILITY or FITNESS FOR A PARTICULAR PURPOSE.  See the
%  GNU Lesser General Public License for more details.
% 
%  You should have received a copy of the GNU Lesser General Public License
%  along with this program.  If not, see <http://www.gnu.org/licenses/>.
    
\documentclass[a4paper,10pt]{article}
\usepackage{listings}
\usepackage{color}
%  This file is part of P4
% 
%  Copyright (C) 1996-2016  J.C. Artés, P. De Maesschalck, F. Dumortier,
%                           C. Herssens, J. Llibre, O. Saleta, J. Torregrosa
% 
%  P4 is free software: you can redistribute it and/or modify
%  it under the terms of the GNU Lesser General Public License as published
%  by the Free Software Foundation, either version 3 of the License, or
%  (at your option) any later version.
% 
%  This program is distributed in the hope that it will be useful,
%  but WITHOUT ANY WARRANTY; without even the implied warranty of
%  MERCHANTABILITY or FITNESS FOR A PARTICULAR PURPOSE.  See the
%  GNU Lesser General Public License for more details.
% 
%  You should have received a copy of the GNU Lesser General Public License
%  along with this program.  If not, see <http://www.gnu.org/licenses/>.
       
%
% TO BE INCLUDED IN THE PREAMBLE OF THE TEX FILE WITH \INPUT{MAPLEDEF}.
%
%

\definecolor{maplebgcolor}{rgb}{0.97,.97,.97}

\lstdefinelanguage{maple}
    {%
    morestring=[b]",
    morecomment=[l]\#,
    sensitive=false,
    morekeywords=[1]{proc,local,global,end,for,from,to,by,do,%
            od,if,else,fi,and,or,then,not,restart,type,return,%
    minus,union,read,save,set,list,member},
    morekeywords=[2]{simplify,normal,expand,expanded,evalf,map,collect,diff,%
            subs,eval,op,nops,coeff,rational,float,even,odd,convert,degree,sign,(,),[,]}}

\lstset{%
    basicstyle=\small,
    keywordstyle=[1]\bfseries,
    keywordstyle=[2]\underbar,
    identifierstyle=\slshape,
    firstnumber=auto,
    numbers=left,
    commentstyle=,
    stringstyle=\ttfamily,
    showstringspaces=false,
    language=maple,
    backgroundcolor=\color{maplebgcolor},
    flexiblecolumns=true
}


\title{P4 Maple routines:\\gcd}
\author{}
\date{}

\setlength\marginparwidth{0cm}
\setlength\marginparsep{0cm}
\setlength\oddsidemargin{2cm}
\setlength\textwidth{\paperwidth}
\addtolength\textwidth{-2\oddsidemargin}
\addtolength\oddsidemargin{-1in}

\begin{document}
\maketitle

\section{Overview}

Here, only one routine is implemented: \verb+my_gcd+, to calculate the greatest common divisor
of two multivariate polynomials.

\section{Implementation}

The routine \verb+my_gcd+ operates on two polynomials, and returns a list of three items.  The first
item is the reduced vector field (the list of two reduced polynomials).  If a non-trivial common
divisor is found, the second item is 1 and the third item is the common divisor.  If no divisor is
found, the second item is 0, and the third item is a dummy.  The routine is called in the very
beginning of the program sequence.

\begin{lstlisting}[name=gcd]

restart;
read( "tools.m" );
read( "num_coeff.m" );

my_gcd := proc( _f, _g )
    local f,g, f2, g2, ff, c, gc, l, r; global rounded;
    f := norpoly2( _f, x, y );
    g := norpoly2( _g, x, y );
    if not rounded then
        f := convert( f, rational );
        g := convert( g, rational );
    fi;
    if f = 0 and ddeg(g,x,y) <> 0 then
        ff := [0,1];
        c := 1;
        gc := g;
    else if g = 0 and ddeg(f,x,y) <> 0 then
        ff := [1,0];
        c := 1;
        gc := f;
    else
        if not(rounded) and num_coeff(f,x,y) and num_coeff(g,x,y) then
            l := lcm( denom(f), denom(g) );
            f2 := f*l;
            g2 := g*l;
            gc := norpoly2( gcd(f2,g2) / l, x, y );
            if ddeg(gc,x,y) <> 0 then
                writef( "The system has a common factor: %a\n", gc );
                c := 1;
                ff := [ norpoly2(f/gc,x,y), norpoly2(g/gc,x,y) ];
                writef( "We shall study the differential system:\n" );
                writef( "    x'=%a\n", ff[1] );
                writef( "    y'=%a\n", ff[2] );
                else
                writef( "The system has no common factors.\n" );
                ff := [f,g];
                c := 0;
            fi;
        else
            r := rounded;
            rounded := true;
            gc := gcd( f, g );
            if ddeg( gc, x, y ) <> 0 then
                writef( "The system has a comman factor: %a\n", gc );
                ff := [ f/gc, g/gc ];
                c := 1;
                writef( "We shall study the differential system:\n" );
                writef( "    x'=%a\n", ff[1] );
                writef( "    y'=%a\n", ff[2] );
            else
                writef( "The system has no common factors.\n" );
                ff := [f,g];
                c := 0;
            fi;
            rounded := r;
        fi;
    fi fi;
    [ ff, c, gc ];
end:
\end{lstlisting}


\subsection{Saving the routines in a library}

The routine is saved in a library "gcd.m".

\begin{lstlisting}[name=gcd]

save( my_gcd, "gcd.m" );

\end{lstlisting}

\end{document}
