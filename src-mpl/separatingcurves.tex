%  This file is part of P4
% 
%  Copyright (C) 1996-2018  J.C. Artés, P. De Maesschalck, F. Dumortier,
%                           C. Herssens, J. Llibre, O. Saleta, J. Torregrosa
% 
%  P4 is free software: you can redistribute it and/or modify
%  it under the terms of the GNU Lesser General Public License as published
%  by the Free Software Foundation, either version 3 of the License, or
%  (at your option) any later version.
% 
%  This program is distributed in the hope that it will be useful,
%  but WITHOUT ANY WARRANTY; without even the implied warranty of
%  MERCHANTABILITY or FITNESS FOR A PARTICULAR PURPOSE.  See the
%  GNU Lesser General Public License for more details.
% 
%  You should have received a copy of the GNU Lesser General Public License
%  along with this program.  If not, see <http://www.gnu.org/licenses/>.

\documentclass[a4paper,10pt]{article}
\usepackage{listings}
\usepackage{color}
%  This file is part of P4
% 
%  Copyright (C) 1996-2016  J.C. Artés, P. De Maesschalck, F. Dumortier,
%                           C. Herssens, J. Llibre, O. Saleta, J. Torregrosa
% 
%  P4 is free software: you can redistribute it and/or modify
%  it under the terms of the GNU Lesser General Public License as published
%  by the Free Software Foundation, either version 3 of the License, or
%  (at your option) any later version.
% 
%  This program is distributed in the hope that it will be useful,
%  but WITHOUT ANY WARRANTY; without even the implied warranty of
%  MERCHANTABILITY or FITNESS FOR A PARTICULAR PURPOSE.  See the
%  GNU Lesser General Public License for more details.
% 
%  You should have received a copy of the GNU Lesser General Public License
%  along with this program.  If not, see <http://www.gnu.org/licenses/>.
       
%
% TO BE INCLUDED IN THE PREAMBLE OF THE TEX FILE WITH \INPUT{MAPLEDEF}.
%
%

\definecolor{maplebgcolor}{rgb}{0.97,.97,.97}

\lstdefinelanguage{maple}
    {%
    morestring=[b]",
    morecomment=[l]\#,
    sensitive=false,
    morekeywords=[1]{proc,local,global,end,for,from,to,by,do,%
            od,if,else,fi,and,or,then,not,restart,type,return,%
    minus,union,read,save,set,list,member},
    morekeywords=[2]{simplify,normal,expand,expanded,evalf,map,collect,diff,%
            subs,eval,op,nops,coeff,rational,float,even,odd,convert,degree,sign,(,),[,]}}

\lstset{%
    basicstyle=\small,
    keywordstyle=[1]\bfseries,
    keywordstyle=[2]\underbar,
    identifierstyle=\slshape,
    firstnumber=auto,
    numbers=left,
    commentstyle=,
    stringstyle=\ttfamily,
    showstringspaces=false,
    language=maple,
    backgroundcolor=\color{maplebgcolor},
    flexiblecolumns=true
}


\title{P4 Maple routines:\\Separating Curves}
\author{}
\date{}

\setlength\marginparwidth{0cm}
\setlength\marginparsep{0cm}
\setlength\oddsidemargin{2cm}
\setlength\textwidth{\paperwidth}
\addtolength\textwidth{-2\oddsidemargin}
\addtolength\oddsidemargin{-1in}

\begin{document}
\maketitle

\section{Overview}

This file is used by P4 to plot the piecewise configuration. Given a number of
curves $(f_r)_{r=1,\dots,N}$, it gives a set of points that are determined by
$f_r(x,y)=0$, within a given rectangle $[x_1,x_2]\times[y_1,y_2]$. The routine
to be called is \emph{findSeparatingCurves}, and afterwards, the value
\emph{returnvalue} iseither $0$ or $1$, depending on whether an error occurred
or not (0=success, 1=failure).

From Maple 10, a compatibility problem has added the need to include a
``SimplifyPlot'' procedure inside the implementation. This changes any occuring
Arrays in the plot structure to lists or double lists.

\section{Implementation}

Default values of the user parameters:

\begin{lstlisting}[name=p4gcf]
restart;
user_numpointscurves := [100]:
user_curves := [x*y]:
user_curvesfile := "untitled_sepcurves.tab":
returnvalue := 0:
\end{lstlisting}

The main routine opens the output file, stores the number of curves, and
executes for every single separating curve the subroutine findSeparatingCurve,
which stores the data of that specific curve in a file.

\begin{lstlisting}[name=separatingcurves]
findAllSeparatingCurves := proc()
    global user_f, user_numpoints, user_curves, rounded, user_numeric,
            returnvalue;
    local fp,k,r,numcurves,numcurveregions;

    user_numeric := true;
    rounded := true;
    numcurves := nops(user_curves);
    numcurveregions := nops(curveregions);
    fp := fopen(user_curvesfile, WRITE, TEXT);

    for r from 1 to numcurves do
        if type(user_curves[r], equation) or type(user_curves[r], `<=`) then
            user_curves[r] := lhs(user_curves[r])-rhs(user_curves[r]);
        end if;
        if CheckIfGoodPolynomial(user_curves[r]) = false then
            printf("The separating curve #%d: %a is not well-defined (not polynomial, or coefficients do not evaluate numerically.)\n", r, user_curves[r]);
            fclose(fp);
            returnvalue := 0;
            NULL;
            return;
        end if;
    end do;

    fprintf(fp, "p5\n%d %d %d %d\n", user_p, user_q, user_precision, numcurves);

    for r from 1 to numcurves do
        user_numpoints := user_numpointscurves[r];
        user_f := user_curves[r];
        if not findSeparatingCurve(fp) then
            fclose(fp);
            fclose(fopen(user_file, WRITE, TEXT)); # clear file
            returnvalue := 0;
            NULL;
            return;
        end if;
    end do;
    fclose(fp);
    returnvalue := 1;
    NULL;
end:
\end{lstlisting}

The routine findSeparatingCurve now prepares the calculations in the several
charts, and calls a routine, similar to the routine to find greatest common
factor-points, to write the data to a file.

\begin{lstlisting}[name=separatingcurves]
findSeparatingCurve := proc(fp)
    global user_x1, user_x2, user_y1, user_y2, u, v, user_f;
    local f, fU1, fV1, fU2, fV2, fC, z1, z2, s, L, j, d;

    f := user_f;    # store original curve
    d := ddeg(f, x, y);
    if d = 0 then
        fprintf(fp, "%d\n", 0)
    else
        L := coeff_degree2(gcff, x, y);
        fprintf(fp, "%d\n", d);
    end if;

    L := coeff_degree2(f, x, y);
    fprintf(fp, "%d", nops(L));
    for j from 1 to nops(L) do
        fprintf(fp, " %d %d %g", op(1, L[j]), op(2, L[j]), evalf(op(3,L[j])));
    end do;
    fprintf(fp, "\n");

    fU1 := subs(x = 1/z2^user_p, y = z1/z2^user_q, f);
    fU1 := norpoly2(fU1, z1, z2);
    fU1 := subs(z1=x, z2=y, numer(fU1));
    L := coeff_degree2(fU1, x, y);
    fprintf(fp, "%d", nops(L));
    for j from 1 to nops(L) do
        fprintf(fp, " %d %d %g", op(1, L[j]), op(2, L[j]), evalf(op(3,L[j])));
    end do;
    fprintf(fp, "\n");

    fU2 := subs(x = z1/z2^user_p, y = 1/z2^user_q, f);
    fU2 := norpoly2(fU2, z1,z2);
    fU2 := subs(z1=x, z2=y, numer(fU2));
    L := coeff_degree2(fU2, x, y);
    fprintf(fp, "%d", nops(L));
    for j from 1 to nops(L) do
        fprintf(fp, " %d %d %g", op(1, L[j]), op(2, L[j]), evalf(op(3,L[j])));
    end do;
    fprintf(fp, "\n");

    fV1 := subs(x = -1/z2^user_p, y = z1/z2^user_q, f);
    fV1 := norpoly2(fV1, z1,z2);
    fV1 := subs(z1=x, z2=y, numer(fV1));
    L := coeff_degree2(fV1, x, y);
    fprintf(fp, "%d", nops(L));
    for j from 1 to nops(L) do
        fprintf(fp, " %d %d %g", op(1, L[j]), op(2, L[j]), evalf(op(3,L[j])));
    end do;
    fprintf(fp, "\n");

    fV2 := subs(x = z1/z2^user_p, y = -1/z2^user_q, f);
    fV2 := norpoly2(fV2, z1,z2);
    fV2 := subs(z1=x, z2=y, numer(fV2));
    L := coeff_degree2(fV2, x, y);
    fprintf(fp, "%d", nops(L));
    for j from 1 to nops(L) do
        fprintf(fp, " %d %d %g", op(1, L[j]), op(2, L[j]), evalf(op(3,L[j])));
    end do;
    fprintf(fp, "\n");

    if user_p = 1 and user_q = 1 then

        user_f := f;
        user_x1 := -1.0;
        user_x2 := 1.0;
        user_y1 := -1.0;
        user_y2 := 1.0;
        u := x;
        v := y;
        printf("Chart R2: ");
        if not findSeparatingCurveInChart(fp) then
            return false;
        end if;

        user_f := fU1;
        user_x1 := -1.0;
        user_x2 := 1.0;
        user_y1 := 0.0;
        user_y2 := 1.0;
        u := x;
        v := y;
        printf("Chart U1: ");
        if not findSeparatingCurveInChart(fp) then
            return false;
        end if;

        user_f := fU2;
        user_x1 := -1.0;
        user_x2 := 1.0;
        user_y1 := 0.0;
        user_y2 := 1.0;
        u := x;
        v := y;
        printf("Chart U2: ");
        if not findSeparatingCurveInChart(fp) then
            return false;
        end if;

        user_f := fU1;
        user_x1 := -1.0;
        user_x2 := 1.0;
        user_y1 := -1.0;
        user_y2 := 0.0;
        u := x;
        v := y;
        printf("Chart V1': ");
        if not findSeparatingCurveInChart(fp) then
            return false;
        end if;

        user_f := fU2;
        user_x1 := -1.0;
        user_x2 := 1.0;
        user_y1 := -1.0;
        user_y2 := 0.0;
        u := x;
        v := y;
        printf("Chart V2': ");
        if not findSeparatingCurveInChart(fp) then
            return false;
        end if;

    else
        fC := subs(x=U/s^user_p, y = V/s^user_q, f);
        fC := norpoly3(fC, s, U, V);
        fC := subs(s=x, numer(fC));

        L := coeff_degree3(fC, x, U, V);
        fprintf(fp, "%d", nops(L));
        for j from 1 to nops(L) do
            fprintf(fp, " %d %d %d %g", op(1, L[j]), op(2, L[j]), op(3, L[j]),
                    evalf(op(4,L[j])));
        end do;
        fprintf(fp, "\n");

        user_f := eval(f,{x=U,y=V});
        user_x1 := 0.0;
        user_x2 := 1.0;
        user_y1 := 0.0;
        user_y2 := evalf(2*Pi);
        u := x*cos(y);
        v := x*sin(y);
        printf("Chart R2 (Lyapunov-weights): ");
        if not findSeparatingCurveInChart(fp) then
            return false;
        end if;

        user_x1 := 0.0;
        user_x2 := 1.0;
        user_y1 := evalf(-Pi/4);
        user_y2 := evalf(Pi/4);
        user_f := fC;
        u := cos(y);
        v := sin(y);
        printf("Chart U1 (Lyapunov-weights): ");
        if not findSeparatingCurveInChart(fp) then
            return false;
        end if;

        user_x1 := 0.0;
        user_x2 := 1.0;
        user_y1 := evalf(Pi/4);
        user_y2 := evalf(Pi-Pi/4);
        user_f := fC;
        u := cos(y);
        v := sin(y);
        printf("Chart U2 (Lyapunov-weights): ");
        if not findSeparatingCurveInChart(fp) then
            return false;
        end if;

        user_x1 := 0.0;
        user_x2 := 1.0;
        user_y1 := evalf(Pi-Pi/4);
        user_y2 := evalf(Pi+Pi/4);
        user_f := fC;
        u := cos(y);
        v := sin(y);
        printf("Chart V1 (Lyapunov-weights): ");
        if not findSeparatingCurveInChart(fp) then
            return false;
        end if;

        user_x1 := 0.0;
        user_x2 := 1.0;
        user_y1 := evalf(-Pi+Pi/4);
        user_y2 := evalf(-Pi/4);
        user_f := fC;
        u := cos(y);
        v := sin(y);
        printf("Chart V2 (Lyapunov-weights): ");
        if not findSeparatingCurveInChart(fp) then
            return false;
        end if;
    end if;
    return true;
end:
\end{lstlisting}

\begin{lstlisting}[name=separatingcurves]
findSeparatingCurveInChart := proc(fp)
    global u, v, returnvalue, facs, fac, user_f, user_numpoints;
    local m, n, P, k, i, j, s, f, CURVES_ARRAY_TO_LISTLIST, SimplifyPlot;

    CURVES_ARRAY_TO_LISTLIST := proc()
        local ss, k;
        ss := [$1..nargs];
        for k from 1 to nargs do
            ss[k] := args[k];
            if type(ss[k], Array) then
                if ArrayNumDims(ss[k]) = 2 then
                    ss[k] := convert(ss[k], listlist);
                elif ArrayNumDims(ss[k]) = 1 then
                    ss[k] := convert(ss[k], list);
                end if;
            end if;
        end do;
        CURVES(op(ss));
    end proc;

    SimplifyPlot := P -> eval(P, CURVES=CURVES_ARRAY_TO_LISTLIST);

    returnvalue := 1;
    if degree(user_f, [x,y,U,V]) <= 0 then
        printf("%a: Level curve is a constant.\n", user_f);
        fprintf(fp, "%d\n", 0);
        return true;
    else
        m := round(sqrt(user_numpoints)) + 1:
        n := m:
        facs := factors(subs({U=u,V=v},user_f))[2];
        P := map(fac -> plots[implicitplot](fac,
                    x=user_x1..user_x2,y=user_y1..user_y2,axes=none,
                    view=[user_x1..user_x2,user_y1..user_y2],grid=[m,n],
                    color=black), facs):
        P := map(SimplifyPlot, P);
        P := map(fac -> op(select(pt -> type(pt,list), [op(op(1,fac))])), P):
        k := 0:
        for i from 1 to nops(P) do
            k := k + nops(P[i]);
        end do;
        fprintf(fp, "%d\n", k);
        k := 0;
        for i from 1 to nops(P) do
            for j from 1 to nops(P[i]) do
                s := sprintf("%g %g ", P[i,j,1], P[i,j,2]);
                fprintf(fp, "%s", s);
                k := k+1;
            end do;
            if i <> nops(P) then
                fprintf(fp, ",\n");
            else
                fprintf(fp, "\n");
            end if;
        end do:
        printf("%a: %d points returned.\n", user_f, k);
    end if;
    returnvalue := 0;
    return true;
end:
\end{lstlisting}

Saving all to a library:

\begin{lstlisting}[name=separatingcurves]
save(returnvalue, findAllSeparatingCurves, findSeparatingCurve,
        findSeparatingCurveInChart, "separatingcurves.m");
\end{lstlisting}

\section{Syntax of output file}

The syntax of the output file is as follows:

\begin{enumerate}
\item
    Identifier P5
\item
    Integers p, q, precision, numcurves
\item
    For each curve:
    \begin{itemize}
    \item
        curve polynomial in (x,y) in finite chart
    \item
        curve polynomial in U1 chart
    \item
        curve polynomial in U2 chart
    \item
        curve polynomial in V1 chart
    \item
        curve polynomial in V2 chart
    \item
        (only when $(p,q)\not=(1,1)$: curve polynomial in cylindrical
        coordinates)
    \item
        Plot data in 5 different charts.
    \end{itemize}
\end{enumerate}

\end{document}
