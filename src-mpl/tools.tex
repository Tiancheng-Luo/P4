%  This file is part of P4
% 
%  Copyright (C) 1996-2017  J.C. Artés, P. De Maesschalck, F. Dumortier,
%                           C. Herssens, J. Llibre, O. Saleta, J. Torregrosa
% 
%  P4 is free software: you can redistribute it and/or modify
%  it under the terms of the GNU Lesser General Public License as published
%  by the Free Software Foundation, either version 3 of the License, or
%  (at your option) any later version.
% 
%  This program is distributed in the hope that it will be useful,
%  but WITHOUT ANY WARRANTY; without even the implied warranty of
%  MERCHANTABILITY or FITNESS FOR A PARTICULAR PURPOSE.  See the
%  GNU Lesser General Public License for more details.
% 
%  You should have received a copy of the GNU Lesser General Public License
%  along with this program.  If not, see <http://www.gnu.org/licenses/>.
      
\documentclass[a4paper,10pt]{article}
\usepackage{listings}
\usepackage{color}
%  This file is part of P4
% 
%  Copyright (C) 1996-2016  J.C. Artés, P. De Maesschalck, F. Dumortier,
%                           C. Herssens, J. Llibre, O. Saleta, J. Torregrosa
% 
%  P4 is free software: you can redistribute it and/or modify
%  it under the terms of the GNU Lesser General Public License as published
%  by the Free Software Foundation, either version 3 of the License, or
%  (at your option) any later version.
% 
%  This program is distributed in the hope that it will be useful,
%  but WITHOUT ANY WARRANTY; without even the implied warranty of
%  MERCHANTABILITY or FITNESS FOR A PARTICULAR PURPOSE.  See the
%  GNU Lesser General Public License for more details.
% 
%  You should have received a copy of the GNU Lesser General Public License
%  along with this program.  If not, see <http://www.gnu.org/licenses/>.
       
%
% TO BE INCLUDED IN THE PREAMBLE OF THE TEX FILE WITH \INPUT{MAPLEDEF}.
%
%

\definecolor{maplebgcolor}{rgb}{0.97,.97,.97}

\lstdefinelanguage{maple}
    {%
    morestring=[b]",
    morecomment=[l]\#,
    sensitive=false,
    morekeywords=[1]{proc,local,global,end,for,from,to,by,do,%
            od,if,else,fi,and,or,then,not,restart,type,return,%
    minus,union,read,save,set,list,member},
    morekeywords=[2]{simplify,normal,expand,expanded,evalf,map,collect,diff,%
            subs,eval,op,nops,coeff,rational,float,even,odd,convert,degree,sign,(,),[,]}}

\lstset{%
    basicstyle=\small,
    keywordstyle=[1]\bfseries,
    keywordstyle=[2]\underbar,
    identifierstyle=\slshape,
    firstnumber=auto,
    numbers=left,
    commentstyle=,
    stringstyle=\ttfamily,
    showstringspaces=false,
    language=maple,
    backgroundcolor=\color{maplebgcolor},
    flexiblecolumns=true
}


\title{P4 Maple routines:\\tools}
\author{}
\date{}

\setlength\marginparwidth{0cm}
\setlength\marginparsep{0cm}
\setlength\oddsidemargin{2cm}
\setlength\textwidth{\paperwidth}
\addtolength\textwidth{-2\oddsidemargin}
\addtolength\oddsidemargin{-1in}

\begin{document}
\maketitle

\section{Overview}

\section{Implementation}

First, let us define some global variables.  The variable \verb+Hardware_precision+ is not
used in the program, except to define \verb+user_precision+.  This is the number of digits
that is used in numeric calculations.  By setting it to the hardware precision, maximum
performance is reached (because the processor's capacity is used in full).  The boolean
variable \verb+rounded+ specifies wether or not to calculate numerically.  This flag reflects
the value of \verb+user_numeric+ (which is defined elsewhere, and which is set by the user
in the P4 program), but may change locally.  For example, when the user specifies symbolic
calculations, Maple will set the variable \verb+rounded+ to true while determining the type
of singularities which could only be determined numerically.

The variables \verb+currentopenfile+ and \verb+openfilelist+ are used during file i/o (see below).

\begin{lstlisting}[name=tools]
restart;
rounded := false:
Hardware_precision := floor(evalhf(Digits)):
user_precision := Hardware_precision:

currentopenfile := {-infinity}:
openfilelist := {}:
\end{lstlisting}

\section{Compatibility routines}

A number of routines have been introduced to copy the behaviour of reduce.  These routines
are the following:

\begin{enumerate}
\item   \verb+reduce_coeff+: returns coefficients w.r.t.~a variable of a multivariate polynomial as a sorted list.
\item   \verb+reduce_llt+: floating-point version of `certainly less than'
\item   \verb+reduce_gt+: floating-point version of `certainly greater than'
\item   \verb+reduce_gteq+: floating-point version of `greater than or equal'
\item   \verb+reduce_eeq+: floating-point version of `equal'
\item   \verb+reduce_nneq+: floating-point version of `certainly not equal'
\item   \verb+reduce_imp+: calculate imaginary part, with some care taken regarding precision
\item   \verb+reduce_conj+: calculate complex conjugated number
\item   \verb+reduce_re+: calculate imaginary part
\item   \verb+reduce_im+: calculate real part
\end{enumerate}

\begin{lstlisting}[name=tools]

reduce_imp := z -> evalf(reduce_im(z)):
reduce_re :=  z -> (z + reduce_conj(z))/2:
reduce_im :=  z -> (z - reduce_conj(z))/(2*I):

reduce_llt :=   (a,b) -> evalb(evalf(a-b) <= -10^(-user_precision)):
reduce_gt :=    (a,b) -> evalb(evalf(a-b) >= 10^(-user_precision)):
reduce_gteq :=  (a,b) -> evalb(evalf(a-b) >= -10^(-user_precision)):
reduce_eeq :=   (a,b) ->  evalb(evalf(abs(a-b)) < 10^(-user_precision)):
reduce_nneq :=  (a,b) -> evalb(evalf(abs(a-b)) >= 10^(-user_precision)):

reduce_evalf := evalf:
reduce_conj :=  conjugate:

norpoly := (P,x) -> if user_simplify then expand(user_simplifycmd(normal(P))) else expand(normal(P)) end if;
norpoly2 := (P,x,y) -> if user_simplify then expand(user_simplifycmd(normal(P))) else expand(normal(P)) end if;
norpoly3 := (P,x,y,z) -> if user_simplify then expand(user_simplifycmd(normal(P))) else expand(normal(P)) end if;

reduce_coeff := proc(f, x)
    local g, g1, g2, j;
    g := norpoly(f,x);
    g1 := numer(g);
    g2 := denom(g);
    if g1 = 0 then [0] else [ coeff(g1, x, j)/g2 $ j = 0..degree(g1, x) ] fi;
end:
\end{lstlisting}

\section{File I/O}

\begin{itemize}
\item   \verb+openfile+: redirects output of writef to given file(s).
\item   \verb+closefile+: closes a file.
\item   \verb+closeallfiles+: close all files.
\item   \verb+writef+: write to screen, or to file.
\item   \verb+write_number+: write a number (for compatibility).
\end{itemize}

These routines keep a list of all files that are open in one hand, and of all files
that are selected for output redirection on the other hand.  For example, at one
moment, Maple may have opened three files A, B and C, and may choose files B, C and the terminal
as the files to which the output is redirected.  Any writef statement will result in a write
to B, C and the terminal.

The variable \verb+openfilelist+ is a set of files that are currently opened (excluding the terminal).
Each member in this set is a list of two items: the filename and the file handle.  The variable
\verb+currentopenfile+ is a set of files that are currently selected for redirection: it is a set
of file handles that will receive the output of all writef instructions.  This set may contain the
handle $-\infty$, which stands for terminal output.  The set is never empty.

\begin{lstlisting}[name=tools]
openfile := proc()
    local kk, k, fh, fname;
    global currentopenfile, openfilelist;
    currentopenfile := {};

    for kk from 1 to nargs do
        fname := eval(args[kk]);
        if fname = terminal then
            currentopenfile := currentopenfile union { -infinity }
        else
            for k from 1 to nops(openfilelist) do
                if op(1,openfilelist[k]) = fname then break fi
            od;
            if k <= nops(openfilelist) then
                currentopenfile := currentopenfile union { op(2,openfilelist[k]) }
            else
                fh := fopen(fname, WRITE, TEXT);
                currentopenfile := currentopenfile union { fh };
                openfilelist := openfilelist union { [ fname, fh ] }
            fi
        fi
    od;
    if nops(currentopenfile)=0 then
        currentopenfile := {-infinity}
    fi;
end:
\end{lstlisting}

The next routine closes a file with given filename.  Once you close a file and choose
to re-open it later, then the previous contents will be deleted! (So, do not close files
until you know that the last bit of information is written to it.)
In the implementation, some care is taken in order to make sure that the files that
are closed are removed from the `redirection' list.

\begin{lstlisting}[name=tools]
closefile := proc(filename)
    local k;
    global currentopenfile, openfilelist;

    if filename <> terminal then
        for k from 1 to nops(openfilelist) do
            if op(1,openfilelist[k]) = filename then
                break
            fi
        od;

        if k <= nops(openfilelist) then
            fclose(op(2,openfilelist[k]));
            if member(op(2,openfilelist[k]), currentopenfile) then
                currentopenfile := currentopenfile minus {op(2,openfilelist[k])};
                if nops(currentopenfile) = 0 then
                    currentopenfile := {-infinity}
                fi;
            fi;
            openfilelist := openfilelist minus { openfilelist[k] }
        fi

    else
        # filename=terminal
        if member(-infinity, currentopenfile) and nops(currentopenfile)>1 then
            currentopenfile := currentopenfile minus {-infinity};
        fi;
    fi;
end:

closeallfiles := proc(filename)
    while nops(openfilelist) <> 0 do
        closefile(op(1, openfilelist[1]))
    od
end:

write_number := proc(n)
    if rounded then
        writef("%g\n", evalf(n))
    else
        writef("%a\n", n) fi;
end:
\end{lstlisting}

The writef procedure reflects the behaviour of the C-routine printf:  it uses
a format string, to determine the format of the variables that need to be
printed, and a variable list of variables.  In the format string, use \verb+%g+ to write
a floating-point number, use \verb+%a+ to write a general mathematical expression (symbolically).
This mathematical expression may be a number, a polynomial, a series, a list, a set or anything else.
Use \verb+\n+ to write an end-of-line.

\begin{lstlisting}[name=tools]
writef := proc(formatstring)
    local j;
    for j from 1 to nops(currentopenfile) do
        if currentopenfile[j] = -infinity then
            printf(formatstring, args[2..nargs])
        else
            fprintf(currentopenfile[j], formatstring, args[2..nargs])
        fi
    od;
end:
\end{lstlisting}

\subsection{Polynomial tools}

The next three routines are used quite often in the program.  They are used to
simplify intermediate (polynomial) results.  It will expand the polynomial
in powers of the given variable.  During numeric computations, it
makes sure all coefficients are numeric, and it will remove coefficients that
are in absolute value smaller than $10^{-k}$, where $k$ is the precision.  During
symbolic computations, it will perform a simplify command on the coefficients
of the polynomial, in the hope of obtaining simpler expressions.  Sometimes however,
the built-in simplify commands do not `see' that coefficients are zero.  Therefore,
a numeric evaluation will be performed to see if they are zero, and if so, they will
be removed.  This is important, for example in finding out the degree of the polynomial.

The user may control wether or not any simplification command is executed in symbolic mode.
The user may set the boolean flag \verb+user_simplify+.  The command itself is also user-definable
in the command \verb+user_simplifycmd+.  These variables should be defined elsewhere.

The last routine \verb+change_poly+ is implemented for compatibility.

In Maple, special care needs to be taken for `long list assignments': if a list of more then 100
elements is used (as is easily the case when considering terms of blow up vector fields or separatrices),
one cannot re-assign individual elements of this list; instead use the map function.

\begin{lstlisting}[name=tools]
optimizepolynomial1 := proc(f, x)
    local g, L, j, k, z;
    global user_precision, rounded;
    g := norpoly(f,x);
    if not rounded then
        g := convert(g, rational)
    else
        g := evalf(g)
    fi;
    L := eval([ 'coeff(g, x, j)' $ j = 0..max(0,degree(g,x)) ]);
    if not(rounded) and user_simplify then
        L := map(user_simplifycmd,L)
    fi;
    L := map(z ->
        if type(evalf(z),numeric) then
            if abs(evalf(z)) <= 10^(-user_precision) then
                0
            else
                z
            end if
        else
            z
        end if, L);
    eval(add(L[k]*x^(k-1), k=1..nops(L)))
end:

optimizepolynomial2 := proc(f, x, y)
    local g, j, L, z;
    global rounded;
    g := norpoly2(f,x,y);
    openfile(terminal);
    writef("up here");
    closefile(terminal);
    if not rounded then
        g := convert(g, rational)
    else
        g := evalf(g)
    fi;
    L := eval([ 'coeff(g, x, j)' $ j = 0..max(0,degree(g,x)) ]);
    L := map(z -> optimizepolynomial1(z, y), L);
    norpoly(eval(add(L[k]*x^(k-1), k=1..nops(L))), x);
end:

optimizevf :=  (f, x, y) ->
    [ optimizepolynomial2(f[1],x,y), optimizepolynomial2(f[2], x, y) ]:

change_poly := proc(f, x, y)
    local g, h, k, a;
    g := norpoly2(f, x, y);
    h := 0;
    while g <> 0 do
        k := nlterm(g, x, y);
        g := g - k;
        a := eval(eval(k, x=1), y=1);
        if abs(evalf(a)) > 10^(-user_precision) then h := h + k fi
    od;
    h
end:

\end{lstlisting}

Using the above simplication routines, we are ready to implement some polynomial tools:
During the implementation, one should realize that the Maple-command \verb+degree+ returns $-\infty$ when
a polynomial is $0$.  Therefore, the case $0$ is implemented differently in most routines.

\begin{itemize}
\item   \verb+first_term+:   Calculate term of lowest degree of a univariate polynomial.
\item   \verb+lleadterm+:    Calculate term of highest degree of a univariate polynomial.
\item   \verb+nlterm+:       Calculate leading term of a multivariate polynomial.
\item   \verb+ddeg+:         Calculate homogeneous degree of multivariate polynomial.
\item   \verb+low_ddeg+:    Calculate minimum homogeneous degree over all monomials of a multivariate polynomial.
\item   \verb+low_deg1+:     Calculate minimum degree of a univariate polynomial.
\item   \verb+nterm+:       Extract homogeneous degree $d$ from a multivariate polynomial.
\item   \verb+quasihom_degree1+: Calculates quasihomogeneous degree of weights $(p,q)$ of a multivariate polynomial.
\item   \verb+quasihom_degree+: Calculates quasihomogeneous degree of a vector field.
\end{itemize}

\begin{lstlisting}[name=tools]
first_term := proc(_f, x)
    local f;
    f := optimizepolynomial1(_f, x);
    if f = 0 then 0 else coeff(f, x, ldegree(f,x))*x^(ldegree(f,x)) fi;
end:

lleadterm := proc(_f, x)
    local f;
    f := optimizepolynomial1(_f, x);
    if f = 0 then 0 else coeff(f, x, degree(f,x))*x^(degree(f,x)) fi;
end:

nlterm := proc(f, x, y)
    local g, h;
    g := norpoly(lleadterm(f, x), y);
    if g <> 0 then
        h := lleadterm(g, y);
        if h <> 0 then h else g fi
    else
        g := norpoly(lleadterm(f, y), x);
        if g <> 0 then g else f fi
    fi
end:

ddeg := proc(_f, x, y)
    local f;
    global rounded;
    f := optimizepolynomial2(_f, x, y);
    if f = 0 then 0 else degree(f, {x,y}) fi
end:

low_ddeg := proc(_f, x, y)
    local f;
    f := optimizepolynomial2(_f, x, y);
    if f = 0 then 0 else ldegree(f, {x,y}) fi
end:

low_deg1 := proc(_f, x)
    local f;
    f := evalf(norpoly(_f), x);
    if f = 0. then 0 else ldegree(f, x) fi
end:

nterm := proc(_f, x, y, d)
    local f, j, g;
    f := optimizepolynomial2(_f, x, y);
    g := 0;
    for j from 0 to d do
        g := g + coeff(coeff(f, x, j), y, d-j)*x^j*y^(d-j);
    od;
    g
end:

quasihom_degree1 := proc(_f, x, y, p, q)
    local _x, _y, f;
    f := optimizepolynomial2(_f, x, y);
    f := subs(x=_x^p, y=_y^q, f);
    degree(f, {_x, _y})
end:

quasihom_degree :=(vf, x, y, p, q) ->
    max(quasihom_degree1(vf[1],x,y,p,q) - p, quasihom_degree1(vf[2],x,y,p,q) - q):
\end{lstlisting}

\subsection{List tools}

The next routine sorts a list of points in the plane.

\begin{lstlisting}[name=tools]
sort_list := s -> sort(s, (sle_a,sle_b) ->
    evalb((sle_a[1] < sle_b[1]) or (sle_a[1] = sle_b[1] and sle_a[2] < sle_b[2]))):
\end{lstlisting}

\subsection{Saving the routines in a library}

All global variables and routines are saved in a library "tools.m".

\begin{lstlisting}[name=tools]

save(rounded, Hardware_precision, user_precision, reduce_coeff, write_number,
      reduce_evalf, reduce_llt, reduce_gt, reduce_gteq, reduce_eeq, reduce_nneq,
      optimizepolynomial1, optimizepolynomial2, first_term, lleadterm, nlterm,
      ddeg, low_ddeg, nterm, low_deg1, optimizevf, quasihom_degree1,
      quasihom_degree, change_poly, sort_list, reduce_imp,
      reduce_conj, reduce_re, reduce_im, openfile, closefile, writef,
      currentopenfile, openfilelist, closeallfiles, norpoly, norpoly2, norpoly3,
      "tools.m");
\end{lstlisting}

\end{document}
