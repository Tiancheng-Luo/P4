%   P4 (Polynomial Planar Phase Portraits) GUI SOURCE CODE
%   Software to study polynomial planar differential systems and represent
%   their phase portrait in several spaces, such as Poincaré sphere. 
%   URL: http://github.com/oscarsaleta/P4-src-gui
%   
%   Copyright (C) 1996-2016  J.C. Artés, C. Herssens, P. De Maesschalck,
%                            F. Dumortier, J. Llibre, O. Saleta
%   
%   This program is free software: you can redistribute it and/or modify
%   it under the terms of the GNU Lesser General Public License as published
%   by the Free Software Foundation, either version 3 of the License, or
%   (at your option) any later version.
%   
%   This program is distributed in the hope that it will be useful,
%   but WITHOUT ANY WARRANTY; without even the implied warranty of
%   MERCHANTABILITY or FITNESS FOR A PARTICULAR PURPOSE.  See the
%   GNU Lesser General Public License for more details.
%   
%   You should have received a copy of the GNU Lesser General Public License
%   along with this program.  If not, see <http://www.gnu.org/licenses/>.
%      
\documentclass[a4paper,10pt]{article}
\usepackage{listings}
\usepackage{color}
%  This file is part of P4
% 
%  Copyright (C) 1996-2016  J.C. Artés, P. De Maesschalck, F. Dumortier,
%                           C. Herssens, J. Llibre, O. Saleta, J. Torregrosa
% 
%  P4 is free software: you can redistribute it and/or modify
%  it under the terms of the GNU Lesser General Public License as published
%  by the Free Software Foundation, either version 3 of the License, or
%  (at your option) any later version.
% 
%  This program is distributed in the hope that it will be useful,
%  but WITHOUT ANY WARRANTY; without even the implied warranty of
%  MERCHANTABILITY or FITNESS FOR A PARTICULAR PURPOSE.  See the
%  GNU Lesser General Public License for more details.
% 
%  You should have received a copy of the GNU Lesser General Public License
%  along with this program.  If not, see <http://www.gnu.org/licenses/>.
       
%
% TO BE INCLUDED IN THE PREAMBLE OF THE TEX FILE WITH \INPUT{MAPLEDEF}.
%
%

\definecolor{maplebgcolor}{rgb}{0.97,.97,.97}

\lstdefinelanguage{maple}
    {%
    morestring=[b]",
    morecomment=[l]\#,
    sensitive=false,
    morekeywords=[1]{proc,local,global,end,for,from,to,by,do,%
            od,if,else,fi,and,or,then,not,restart,type,return,%
    minus,union,read,save,set,list,member},
    morekeywords=[2]{simplify,normal,expand,expanded,evalf,map,collect,diff,%
            subs,eval,op,nops,coeff,rational,float,even,odd,convert,degree,sign,(,),[,]}}

\lstset{%
    basicstyle=\small,
    keywordstyle=[1]\bfseries,
    keywordstyle=[2]\underbar,
    identifierstyle=\slshape,
    firstnumber=auto,
    numbers=left,
    commentstyle=,
    stringstyle=\ttfamily,
    showstringspaces=false,
    language=maple,
    backgroundcolor=\color{maplebgcolor},
    flexiblecolumns=true
}


\title{P4 Maple routines:\\weakfocus}
\author{}
\date{}

\setlength\marginparwidth{0cm}
\setlength\marginparsep{0cm}
\setlength\oddsidemargin{2cm}
\setlength\textwidth{\paperwidth}
\addtolength\textwidth{-2\oddsidemargin}
\addtolength\oddsidemargin{-1in}

\begin{document}
\maketitle

\section{Overview}

Implements the weak focus case, calculates lyapunov constants (by making a call to an external C program).
Four routines are exported:
\begin{itemize}
\item   \verb+weak_focus+: describes a weak focus singularity.
\item   \verb+lyapunov+: Calculate lyapunov constants, by making a call to an external C program
\item   \verb+readlyapunovresult+:
\item   \verb+preparelyapunovfile+:
\end{itemize}

\section{Implementation}

\begin{lstlisting}[name=weakfocus]
restart;
read( "tools.m" );
read( "writelog.m" );

weak_focus := proc(f, x0, y0, chart)
    global hamiltonian;
    local stype, ff, aa10, aa01, bb10, s, g, _x, gg, lyp;

    writef( "(%a,%a) is a weak focus.\n", x0, y0 );
    openfile( result_file );
    writef( "(%a,%a) is a weak focus.\n", x0, y0 );
    openfile( terminal );
    if hamiltonian then
        writef( "Since the system is hamiltonian, we have a center.\n" );
        openfile( result_file );
        writef( "Since the system is hamiltonian, we have a center.\n" );
        openfile( terminal );
        stype := 4;
    else
        # move to the origin
        # make a transformation of the form
        #   xx = (-b10*x + a10*y)/sqrt(-a01*b10-a10^2);
        #   yy = y
        #   tt = -t * sqrt(-a01*b10-a10^2);

        ff := translation( f, x, y, x0, y0 );
        aa10 := coeff( coeff( ff[1], x, 1 ), y, 0 );
        aa01 := coeff( coeff( ff[1], x, 0 ), y, 1 );
        bb10 := coeff( coeff( ff[2], x, 1 ), y, 0 );
        s := sqrt( -aa01*bb10-aa10^2 );
        g := [ bb10*subs({x=-s*_x/bb10+aa10*y/bb10}, ff[1])/s^2
               - aa10*subs({x=-s*_x/bb10+aa10*y/bb10},ff[2])/s^2,
               - subs( {x=-s*x/bb10+aa10*y/bb10}, ff[2])/s ];
        g := evalf( subs( _x=x, g ) );
        writef( "The local system: %a\n", g );
        gg := subs( {x=(z+w)/2,y=-I*(z-w)/2}, g[1] ) + I * subs( {x=(z+w)/2,y=-I*(z-w)/2}, g[2] );
        gg := expand(gg);
        if save_all then
            openfile( result_file );
            writef( "z' = %a\n", gg );
            openfile( terminal );
        fi;
        if weakness_level = 0 then
            openfile( result_file, terminal );
            writef( "Lyapunov constants are not calculated. (weakness level=0)\n" );
            stype := 0;
        else
            if user_numeric then
                lyp := lyapunov( gg - I*z, z, w );
            else
                lyp := lyapunovsym( gg - I*z, z, w );
            end if;
            writef( "Lyapunov information: %a\n", lyp );
            if lyp[1] <> 0 then
                openfile( result_file, terminal );
                writef( "The order of weakness is %d\n", lyp[2] );
                writef( "Lyapunov constant is %g\n", -lyp[1] );
                openfile( terminal );
                if lyp[1] < 0 then
                    writef( "We have an unstable singularity.\n" );
                    openfile( result_file );
                    writef( "We have an unstable singularity.\n" );
                    openfile( terminal ); stype := 1;
                else
                    writef( "We have a stable singularity.\n" );
                    openfile( result_file );
                    writef( "We have a stable singularity.\n" );
                    openfile( terminal ); stype := -1;
                fi;
            else
                if lyp[3] = 2 then
                     writef( "We have a center (the system is quadratic).\n" );
                     openfile( result_file );
                     writef( "We have a center (the system is quadratic).\n" );
                     openfile( terminal );
                     stype := 4;
                else
                    if lyp[4] then
                        writef( "We have a center (the system is linear+cubic).\n" );
                        openfile( result_file );
                        writef( "We have a center (the system is linear+cubic).\n" );
                        openfile( terminal );
                        stype := 4;
                    else
                        writef( "The order of weakness is at least %d.\n", weakness_level+1 );
                        openfile( result_file );
                        writef( "The order of weakness is at least %d.\n", weakness_level+1 );
                        openfile( terminal );
                        stype := 0;
                    fi;
                 fi;
            fi;
        fi;
   fi;
   openfile( result_file );
   writef( "########################################################\n" );
   openfile( terminal );
   writef( "########################################################\n" );
   write_weak_focus( x0, y0, chart, stype );
end:


lyapunov := proc( f, z, w )
   local a, bindir, infile, outfile, exefile, result, rndnum, wl, linfo, aux;
   global user_exeprefix, user_tmpdir, user_lypexe, user_lypexe_mpf, user_precision0, user_bindir, user_removecmd;

   rndnum := irem(rand(),1000);
   infile := cat( user_tmpdir, "P4LYP_", rndnum, ".INP" );
   outfile := cat( user_tmpdir, "P4LYP_", rndnum, ".OUT" );
   if user_precision0 = 0 then
        exefile := cat( user_bindir, user_lypexe );
   else
        exefile := cat( user_bindir, user_lypexe_mpf );
   end if;
   linfo := preparelyapunovfile( infile, f, z, w );
   wl := linfo[1];
   writef( "Determining lyapunov constants ...\n" );
   #writef( cat( user_exeprefix, "", exefile, " \"", infile, "\" \"", outfile, "\" MAPLE ",
   #    user_platform, " \"", user_sumtablepath, "\"" ) );
   aux:=ssystem( cat( user_exeprefix, "", exefile, " \"", infile, "\" \"", outfile, "\" MAPLE ",
        user_platform, " \"", user_sumtablepath, "\"" ) );
   #writef(aux[2]);
   result := readlyapunovresult( outfile, wl );
   #aux:=ssystem( cat( user_removecmd, " \"", infile, "\"" ) );
   #writef(aux[2]);
   #aux:=ssystem( cat( user_removecmd, " \"", outfile, "\"" )  );
   #writef(aux[2]);
   [ op(result), linfo[2], linfo[3] ];
end:

readlyapunovresult := proc( filename, weaklevel )
    local lypdata, Vdata, j, n, wval, lypval, okval;

    #writef(cat("reading ",filename));
    lypdata := readdata( filename, 1 );
    writef( "lyapunov data = %a\n", lypdata );
    Vdata := []; n := 1;
    openfile( terminal, result_file );
    for j from 1 to weaklevel do
        if lypdata[n] <> -1 then
            writef( "V(%d)=%g\n", round(lypdata[n]), lypdata[n+1] );
        end if;
        Vdata := [ op(Vdata), [round(lypdata[n]), lypdata[n+1]] ];
        if round(lypdata[n]) = -1 then
            n := n+1; break;
        end if;
        n := n + 2;
    od;
    if round(lypdata[n]) = -1 then
        n := n+1;
    end if;
    okval := round(lypdata[n]); wval := round(lypdata[n+1]); lypval := lypdata[n+2];
    openfile( terminal );
    [lypval,wval];
end:

preparelyapunovfile := proc( filename, _f, z, w )
    local f, cubic, L, d, a, j, wl;

    f := optimizepolynomial2( _f, z, w );
    cubic := false;
    openfile( filename );
    writef( "%d\n", user_precision0*2+2 );  # FROM DECIMAL TO BINARY
    d := ddeg( f, z, w );
    if d = 2 then wl := 3;
        writef( "3\n" )   # we have a quadratic system
    else
        if d = 3 then wl := 5;
           if coeff(coeff(f,z,2),w,0) = 0 and
              coeff(coeff(f,z,0),w,2) = 0 and
              coeff(coeff(f,z,1),w,1) = 0 then
              writef( "5\n" );
              cubic := true;
           else
              wl := weakness_level;
              writef( "%d\n", weakness_level );
           fi;
        else
            wl := weakness_level;
            writef( "%d\n", weakness_level );
    fi; fi;
    writef( "%g\n", 10^(-user_precision) );
    L := coeff_degree2( f, z, w );
    writef( "%d\n", nops(L) );
    for j from 1 to nops(L) do
        a := L[j];
        writef( "  %d %d %g %g\n", a[1], a[2], evalf(Re(a[3])), evalf(Im(a[3])) );
    od;
    closefile( filename );
    [ wl, d, cubic ];
end:
\end{lstlisting}

\begin{lstlisting}[name=weakfocus]
sumtable := proc(n)
    local L,r,s,st_compare;
    #writef( "producing sum-table of order %d...\n", n );
    L:=piecewise(n=0,[[1]],map(op,map(combinat[permute],combinat[partition](n,n))));
    st_compare := ( r,s ) ->
     if r[1] < s[1] then true
     else if r[1]=s[1] then st_compare( r[2..-1], s[2..-1] ) else false fi fi;
     L:=sort(L, st_compare );
    #writef( "finished.  sum-table has %d components.\n", nops(L) );
    return L;
end:
\end{lstlisting}

\begin{lstlisting}
preparelyapunovsym := proc( f, z, w )
    local cubic, L, d, a, j, wl;

    cubic:=false;
    d := ddeg( f, z, w );
    if d = 2 then wl := 3;
    else
        if d = 3 then wl := 5;
           if coeff(coeff(f,z,2),w,0) = 0 and
              coeff(coeff(f,z,0),w,2) = 0 and
              coeff(coeff(f,z,1),w,1) = 0 then
              cubic := true;
           else
              wl := weakness_level;
           fi;
        else
            wl := weakness_level;
    fi; fi;
    L := coeff_degree2( f, z, w );
    [ wl, d, cubic ];
end:
\end{lstlisting}

\begin{lstlisting}[name=weakfocus]
lyapunovsym := proc(_f,z,w)
    # should return [0,0,...] in case nothing is found or [d,V,...] in other case
    local f, linfo, wl, k, T, V, result, ok;
    f := optimizepolynomial2( _f, z, w );

   linfo := preparelyapunovsym( f, z, w );
   wl := linfo[1];
   result := []; # list of lists each element is  a list of 2: [d,V(d)], goes from 1 to weakness level

   ok := false;
    openfile( terminal, result_file );
   for k from 1 to wl do
        T := sumtable(2*k);
        V := 0;
        for j from 1 to nops(T) do
            V := V - part_lyapunov_coeff( f,z,w, T[j], 2*k+1 );
        end do;
        result := [op(result), [V,k]];
        writef( "k=%d V = %a = %f\n", k, V, evalf(V) );
        if reduce_nneq(V,0) then
            #writef( "order of weakness = %d\n", k );
            ok := true;
            break;
        end if;
   end do;
   openfile( terminal );
   if ok then
        result := result[-1];
   else
        result := [0,0];
   end if;
   [ op(result), linfo[2], linfo[3] ];
end:
\end{lstlisting}

\begin{lstlisting}[name=weakfocus]
part_lyapunov_coeff := proc (vecfield,z,w,L,k)
    local R,expr,f,v,ww,i;
    # we first make a list of all polynomials of degrees L[j] listed in L

    R := map(expr->lyp_findpoly( vecfield, z,w,expr+1 ), L );
    if not member(0,R) then
        #writef( "k = %d\n", k );
        #for i from 1 to nops(R) do
            #writef( "R[%d] = %a\n", i-1, R[i] );
        #end do;
        f:=-1;
        #writef( "f = %a\n", f);
        for i from 1 to nops(R)-1 do
            if type(i-1,odd) then
                f:=lyp_Regz(f,R[i],z,w);
            else
                f:=lyp_Imgz(f,R[i],z,w);
            end if;
            #writef( "f = %a\n", f);
        end do;
        i := nops(R);
        if type(i-1,odd ) then
            f := lyp_multc_poly(f,0,-1,z,w);
            f,v := op(lyp_LL(f,R[i],z,w,(k-1)/2));
            #writef( "f = %a\n", f);
            ww := simplify(Im(v));
        else
            f,v := op(lyp_LL(f,R[i], z,w,(k-1)/2));
            #writef( "f = %a\n", f);
            ww := simplify(Re(v));
        end if;
    else
        ww := 0;
    end if;
    #writef( "w = %g\n", ww );
    return ww;
end proc;
\end{lstlisting}

\begin{lstlisting}[name=weakfocus]
lyp_LL := proc(f,g,z,w,n)
    local r;
    r := diff(expand(f*g),z);
    [r,coeff(coeff(r,z,n),w,n)];
end proc;
\end{lstlisting}

\begin{lstlisting}[name=weakfocus]
lyp_G := proc(f,z,w)
    local k, fk, r, l;
    r := 0;
    if degree(f,z) >= 0 then
        for k from 0 to degree(f,z) do
            fk := coeff(f,z,k);
            if degree(fk,w) >= 0 then
                for l from 0 to degree(fk,w) do
                    if k <> l then
                        r := r + coeff(fk,w,l) * 2/(k-l)  * z^k*w^l;
                    else
                        r := r + coeff(fk,w,l) * z^k*w^l;
                    end if;
                end do;
            end if;
        end do;
    end if;
    expand(r);
end proc;
\end{lstlisting}

\begin{lstlisting}[name=weakfocus]
lyp_Regz := proc(f,g,z,w)
    local r;
    r := expand(f*g);
    r := lyp_multc_poly(r,0,-1,z,w);
    r := diff(r,z);
    r := lyp_G(r,z,w);
	r := r + lyp_conj_poly(r,z,w);
    return lyp_multc_poly(r,1/2,0);
end proc;
\end{lstlisting}

\begin{lstlisting}[name=weakfocus]
lyp_Imgz := proc(f,g,z,w)
    local r;
    r := expand(f*g);
    r := diff(r,z);
    r := lyp_G(r,z,w);
	r := r - lyp_conj_poly(r,z,w);
    return lyp_multc_poly(r,0,-1/2);
end proc;
\end{lstlisting}

\begin{lstlisting}[name=weakfocus]
lyp_findpoly := proc(vf,z,w,n)
    local r,k;
    r:=0;
    for k from 0 to n do
        r := r + coeff(coeff(vf,z,k),w,n-k)*z^k*w^(n-k);
    end do;
    r;
end proc;

\end{lstlisting}

\begin{lstlisting}[name=weakfocus]
lyp_multc_poly := proc( f,a,b,z,w )
    return expand( (a+I*b)*f );
end proc;

lyp_conj_poly := proc( f,z,w ) # take conjugate of coefficients and exchange z with w (= zbar)
    local zb;
    expand(conjugate(f)) assuming z::real, w::real;
    eval(eval(eval(%,{z=zb}),w=z),zb=w);
end proc;
\end{lstlisting}

\subsection{Saving the routines in a library}

\begin{lstlisting}[name=weakfocus]

save( weak_focus, lyapunov, readlyapunovresult, preparelyapunovfile,
      lyapunovsym,preparelyapunovsym, sumtable, part_lyapunov_coeff,
      lyp_multc_poly,lyp_findpoly,lyp_Regz,lyp_Imgz,lyp_LL,lyp_G,lyp_conj_poly,
      "weakfocus.m" );

\end{lstlisting}

\end{document}
